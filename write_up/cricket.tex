\documentclass[11pt]{article}
\usepackage{color}
\usepackage[usenames,dvipsnames,svgnames,table]{xcolor}
\definecolor{dark-red}{rgb}{0.7,0.1,0.1} 
\definecolor{dark-blue}{rgb}{0,0,0.7} 
\usepackage[linkcolor=dark-red,
            colorlinks=true,
            urlcolor=dark-blue,
            pdfstartview={XYZ null null 1.00},
            pdfauthor={Gaurav Sood, gsood07@gmail.com; Derek Willis, dwillis@gmail.com},
            citecolor=dark-red,
            bookmarks=false,
            pdfborder={0 0 0},
            pdftitle={Fairly Random}]{hyperref}
            
\usepackage{amsfonts,amssymb,amsbsy,amsmath,amsxtra}

\usepackage{indentfirst}
\usepackage{setspace} % To set line spacing
\usepackage{multirow}

\usepackage{verbatim}

\usepackage[multiple]{footmisc}
\usepackage{fancyvrb}

\usepackage{longtable}

\usepackage[margin=1in]{geometry}
\usepackage{graphicx}

\raggedright
\parindent=1.5em % <- or whatever indent you want

\begin{comment}
	
	tools::texi2dvi("cricket.tex", pdf=TRUE, clean=TRUE)	

\end{comment}

\begin{document}
\title{\vspace{-2.0cm}\normalsize{\textbf{Fairly Random: Impact of Winning the Toss on the Probability of Winning}}}
\author{Gaurav Sood\thanks{\href{mailto:gsood07@gmail.com}{\texttt{gsood07@gmail.com}}.} \and Derek Willis\thanks{\href{mailto:dwillis@gmail.com}{\texttt{dwillis@gmail.com}}.}}
\maketitle
\doublespacing

In many cricket matches, it is claimed that there is a clear advantage to bowling (batting) first. The advantage is pointed to by commentators, and by captains of the competing teams in the pre-toss interview. And by the losing captain in the post-match interview.

The opportunity to bowl or bat first is decided by a coin toss. While this method of deciding on who is advantaged is fair on average, the system isn't fair in any one game. At first glance, the imbalance seems inevitable. After all someone has to bat first. One can, however, devise a baseball like system where short innings are interspersed. If that violates the nature of the game too much, one can easily create pitches that don't deteriorate appreciably over the course of a game. Or, one can come up with an estimate of the advantage and adjust scores accordingly (something akin to an adjustment issued when matches are shortened due to rain).

But before we move to seriously consider these solutions, we may ask about the evidence.

Data are from 43,185 first-class men's cricket matches. It is a virtual census of the relevant population. We have data from matches in all formats.\footnote{For scripts used to scrape the data and the final data set, go \href{http://github.com/soodoku/get-cricket-data/}{here}.} Of the matches, we do not have data on the toss for 2,807 or roughly 7\% of the matches. The primary reason for this is match being abandoned without play. We exclude data from these matches. 

In limited over cricket, a minimum number of overs must be bowled to establish a result. In a One Day match, for instance, each side must bat at least 20 overs. In 706 matches, or roughly 1.7\% of the remaining matches, not enough overs were bowled to get a result. We exclude these matches from our analysis. This leaves us with a final tally of 39,672 matches. 

In cricket matches, three results are possible -- a draw, a win, or a loss. In limited overs cricket, a win or a loss are the most common outcomes. In the first part of our analysis, we focus on winning or losing. We estimate the difference in probability of winning and losing. The team that wins the toss wins the match 2.3\% more often than lose it. This effect, however, is highly variable by format, by conditions, and by whether or not a particular formula was used to adjust scores when it rained. In particular, it is often claimed that toss is more crucial in day and night matches, due to dew and lower visibility of the white ball under lights. And it is often claimed that the toss is more important in tests than one-day matches. And data show as much. 

Thus, counter to intuition, the effect of winning the toss is, on average, at best minor. This may be so because it is impossible to predict well in advance the advantage of bowling or batting first. Or it may simply be because teams are bad at predicting it, perhaps because they use bad heuristics.

\begin{figure}[htbp]
\centering
\caption{Proportion of Matches Won Minus Matches Lost After Winning the Toss}
\includegraphics[scale=.75]{../figs/winbyType.pdf}
\label{fig:summary}
\end{figure}

\begin{figure}[htbp]
\centering
\caption{Proportion of Matches Won Minus Matches Lost After Winning the Toss}
\includegraphics[scale=.75]{../figs/winbyDayNight.pdf}
\label{fig:summary}
\end{figure}


\begin{figure}[htbp]
\centering
\caption{Proportion of Matches Won Minus Matches Lost After Winning the Toss}
\includegraphics[scale=.75]{../figs/winbyDL.pdf}
\label{fig:summary}
\end{figure}

\end{document}