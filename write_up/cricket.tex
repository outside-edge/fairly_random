\documentclass[11pt]{article}
\usepackage{color}
\usepackage[usenames,dvipsnames,svgnames,table]{xcolor}
\definecolor{dark-red}{rgb}{0.7,0.1,0.1} 
\definecolor{dark-blue}{rgb}{0,0,0.7} 
\usepackage[linkcolor=dark-red,
            colorlinks=true,
            urlcolor=dark-blue,
            pdfstartview={XYZ null null 1.00},
            pdfauthor={Gaurav Sood, gsood07@gmail.com; Derek Willis, dwillis@gmail.com},
            citecolor=dark-red,
            bookmarks=false,
            pdfborder={0 0 0},
            pdftitle={Fairly Random}]{hyperref}
            
\usepackage{amsfonts,amssymb,amsbsy,amsmath,amsxtra}

\usepackage{indentfirst}
\usepackage{setspace} % To set line spacing
\usepackage{multirow}

\usepackage{verbatim}

\usepackage[multiple]{footmisc}
\usepackage{fancyvrb}

\usepackage{longtable}

\usepackage[margin=1in]{geometry}
\usepackage{graphicx}

\raggedright
\parindent=1.5em % <- or whatever indent you want

\begin{comment}

setwd(paste0(githubdir, "/cricket-stats/write_up/"))	
tools::texi2dvi("cricket.tex", pdf=TRUE, clean=TRUE)	
setwd(paste0(githubdir, "/cricket-stats"))

\end{comment}

\begin{document}
\title{\vspace{-1.5cm}\normalsize{\textbf{Fairly Random: Impact of Winning the Toss on the Probability of Winning}}}
\author{Gaurav Sood\\\small{\href{mailto:gsood07@gmail.com}{\tt{gsood07@gmail.com}}} \and Derek Willis\\\small{\href{mailto:dwillis@gmail.com}{\tt{dwillis@gmail.com}}}}
\maketitle
\doublespacing

In nearly all cricket matches, it is claimed that there is a clear advantage to bowling (batting) first. The advantage is pointed to by commentators, by team captains in the pre-toss interview, and by the captain of the losing team in the post-match interview. This isn't to say that there is actually an advantage to winning the toss. For it may be impossible to predict well in advance the advantage of bowling or batting first. Or it may be that teams squander the potential advantage by using bad heuristics to choose what they do.

The opportunity to choose whether to bowl or bat first is decided by a coin toss. While this method of granting advatange is fair on average, the system isn't fair in any one game. At first glance, the imbalance seems inevitable. After all, someone has to bat first. One can, however, devise a baseball like system where short innings are interspersed. If that violates the nature of the game too much, one can easily create pitches that don't deteriorate appreciably over the course of a game. Or, one can come up with an estimate of the advantage and adjust scores accordingly (something akin to an adjustment issued when matches are shortened due to rain).

But before we move to seriously consider these solutions, we may ask about the evidence.

Data are from 43,185 first-class men's cricket matches.\footnote{You can find the scripts used to scrape the data, the final data set, and scripts used for analysis at: \href{http://github.com/soodoku/get-cricket-data/}{http://github.com/soodoku/get-cricket-data/}.} It is a near census of the relevant population. We have data on all types of matches: domestic and international Twenty20s -- T20 and T20I respectively, domestic and international one-dayers -- List A and One-Day Internationals (ODI) respectively, and domestic and international multi-day matches -- First Class (FC) and Tests respectively.\footnote{There is a rich variety of first-class matches. In English county cricket, first-class matches last four days. Some first class matches last just a day. Others two days. Yet others three days. And till a particular point in history, a test match lasted as long as it was needed to finish a game. We elide over such differences.} 

Of these matches, we do not have data on the toss for 2,807, or roughly 7\%, of the matches. The primary reason we don't have data on these matches is because the match was abandoned without play. We exclude data from these matches. 

In limited overs cricket, a minimum number of overs must be bowled to establish a result. In a one-day match, for instance, each side must bat at least 20 overs for a result to be declared. In 706 matches, or roughly 1.7\% of the remaining matches, not enough overs were bowled to get a result. We again exclude these matches from our analysis. This leaves us with 39,672 matches. We analyse these data.

In cricket matches, three results are possible -- a draw, a win, or a loss. We look at draws -- a common outcome in matches in which overs aren't limited -- later on. For now, we focus on winning and losing. We elide over differential number of draws across formats, estimating the difference in probability of winning and losing.  

A caveat about interpretation before we discuss the results. As we allude to at the start, it is hard, if not impossible, to distinguish between lack of advantage of winning a toss and not capitalizing on the potential advantage. 

The team that wins the toss wins the match 2.3\% more often than it loses it. This is a reasonable sized advantage -- though likely much smaller than the number that most commentators carry in their heads -- in a competitive sport. This advantage, however, is highly variable by format, by conditions, by whether or not a particular formula was used to adjust scores when it rained, and how much better the team that won the toss is vis-\`{a}-vis the competing team. Some of the variability is expected, but as we will see, expectations are often dashed. 

The conventional wisdom among the lay cricket followers is that toss grants the greatest advantage in multi-day affairs like tests and first class matches, followed by day long affairs, and Twenty20s respectively. And there is good dose of common sense behind the conventional wisdom. Pitches invariably deteriorate over multiple days and batting last in a test match is often challenging. The pitch deteriorates far less over the course of the day, or in case of Twenty20s -- a few hours. And indeed unlimited over matches provide the most consistent advantage -- the average advantage over FC and test matches is north of 2.2\% (see Figure~\ref{type}). But the heftiest advtange is in domestic one-day matches (List A), approximately 4\%. The number for ODIs, however, is a bit puzzling -- there is a slight disadvantage to winning the toss. That can best be ascribed to choosing badly.      

\begin{figure}[htbp]
\centering
\caption{Percentage of Matches Won Minus Matches Lost After Winning the Toss by Type of Match}
\includegraphics[scale=.85]{../figs/winbyType.pdf}
\label{fig:type}
\end{figure}

But type of matches cover only major source of variation and theorizing about the advantage granted by the toss. It is often claimed that the toss is more crucial in day and night matches. Due to dew -- it is thought to make bowling hard, and lower visibility of the white ball under lights -- it allegedly makes catching hard, it is thought that the team that fields second is disadvantaged. It turns out that the conventional wisdom is largely vindicated, except in the case of Twenty20 Internationals (see Figure ~\ref{dn}). In each, domestic one-day and Twenty20 matches, and one-day internationals, the advantage of winning the toss is at least 3\%, and in case of one-day internationals, nearly 8\%.

\begin{figure}[htbp]
\centering
\caption{Percentage of Matches Won Minus Matches Lost After Winning the Toss By Day/Night}
\includegraphics[scale=.85]{../figs/winbyDayNight.pdf}
\label{fig:dn}
\end{figure}


\begin{figure}[htbp]
\centering
\caption{Percentage of Matches Won Minus Matches Lost After Winning the Toss by Duckworth Lewis}
\includegraphics[scale=.85]{../figs/winbyDL.pdf}
\label{fig:dl}
\end{figure}

\begin{figure}[htbp]
\centering
\caption{Percentage of Matches Won Minus Matches Lost After Winning the Toss by Difference in Ranks}
\includegraphics[scale=.85]{../figs/winbyRank.pdf}
\label{fig:ranks}
\end{figure}

\end{document}